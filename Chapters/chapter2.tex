\addcontentsline{toc}{chapter}{Client C$\#$}

\chapter{Client C$\#$}
%%%%%%%%%%%%%%%%%%%%%%%%%%%%%%%%%%%%%%%%%%%%%%%%%%%%%%%%%%%%%%%%%%%%%%%%%%%%%%%%%%%%%%%%%%%%%%%%%%%%%%%%%%%%%%%%%%%%%%%%
\section{Struttura Generale}

Per l'implementazione si è cercato di seguire le best practices, dettate dalle linee guida Microsoft, per la stesura del codice.
Sono stati adottati, come pattern di programmazione, dependency injection\footnote{
    \textbf{Dependency injection}\cite{dependency-injection} è un pattern di programmazione che prevede la risoluzione automatica delle dipendenze delle classi 
    (può essere usata direttamente nel costruttore o nei metodi) tramite un injector (inversion of control).

        \textbf{Inversion of control}\cite{inversion-of-control} è una macro categoria di pattern (ingloba al suo interno anche dependency injection)
        secondo cui, in certi punti del codice, si riceve il controllo di determinate funzioni di un framework.
}, MVC\footnote{
    \textbf{Model-View-Controller (MVC)}\cite{mvc} è un pattern di programmazione che prevede un architettura multi-tier con una netta divisione fra modelli, 
    business logic e interfaccia utente (in questo caso praticamente assente).
} (per quanto possibile) e TDD, appoggiandosi a librerie esterne Nuget\footnote{
    \textbf{Nuget}\cite{nuget} è un package manager per .NET.
} quali \code{Newtonsoft.Json}\cite{newtonsoft-json} e \code{SimpleInjector}\cite{simple-injector}. 

Si è cercato di adottare il più possibile i dettami della programmazione asincrona e di rendere il client thread-safe.

La solution\footnote{
    \textbf{Solution} è un file, con estensione \code{.sln}, che serve per raggruppare più progetti C$\#$ in un unico macro contenitore.  
} è stata partizionata in vari progetti; il progetto \code{Business} contiene la business-logic 
(comprendente servizi, metodi estensione, l'implementazione del client e un wrapper del \code{Container} di \code{SimpleInjector}) e utilizza i modelli di \code{Domain}.

In \code{Test} sono contenute le classi di testing e le fixture \code{XUnit}\footnote{
    \textbf{XUnit}\cite{xunit-repo} è una libreria Nuget\cite{xunit} che mette a disposizione un tool per eseguire unit test.     
}, mentre in \code{Documentation} sono presenti i file generati dal tool Wyam\footnote{
    \textbf{Wyam}\cite{wyam} è un tool che permette l'esportazione della documentazione del codice C$\#$ 
    (sottoforma di tag xml\cite{code-doc}) in pagine html.
} per la documentazione.

%%%%%%%%%%%%%%%%%%%%%%%%%%%%%%%%%%%%%%%%%%%%%%%%%%%%%%%%%%%%%%%%%%%%%%%%%%%%%%%%%%%%%%%%%%%%%%%%%%%%%%%%%%%%%%%%%%%%%%%%
\section{Il Client}

Dalla documentazione ufficale\cite{dandelion-doc} delle API di Dandelion si evince che ogni end-point richiede uno o più testi da analizzare ed una serie di parametri, 
alcuni obbligatori ed altri opzionali; 
pertanto sono stati creati dei modelli, coerenti con tali definizioni, che, passati in argomento a dei servizi, ritornano i DTO delle risposte degli end-point di Dandelion.

Ogni servizio controlla che i parametri inseriti dall'utente rispettino i constraints definiti nella documentazione e costruisce, tramite i metodi \code{ContentBuilder()} e  
\code{UriBuilder()}, un dizionario di parametri e l'URI che identifica l'end-point. 

Infine il servizio chiama il metodo generico \code{CallApiAsync()}, passandogli il content, la URI ed il metodo HTTP; 
nel metodo di chiamata si è scelto di gestire separatamente i metodi HTTP GET e DELETE dato che essi non permettono il passaggio parametri nel boby 
della chiamata. Pertanto è stato necessario inviare il content come query parameter facendone il percent-encoding.

Dandelion ammette sia chiamate GET che POST ai servizi, tuttavia nel caso del metodo GET non è garantito il corretto 
funzionamento del servizio per testi che superano i 2000 caratteri; pertanto nel caso in cui l'utente non specifichi il metodo HTTP, di default verrà
usato il metodo POST.

Se la chiamata è andata a buon fine il JSON di ritorno viene mappato in un DTO specifico, a seconda del servizio scelto, e restituito direttamente all'utente.

\begin{lstlisting}[style=CSharpStyle, caption=Metodo generico del client per le chiamate HTTP]
public Task<T> CallApiAsync<T>(string uri, List<KeyValuePair<string, string>> content, HttpMethod method = null)
{
    var result = new HttpResponseMessage();
    if (method == null)
    {
        method = HttpMethod.Post;
    }

    if (_client.BaseAddress == null)
    {
        _client.BaseAddress = new Uri(Localizations.BaseUrl);
    }

    return Task.Run(async () =>
    {
        if (method == HttpMethod.Get)
        {
            string query;
            using (var encodedContent = new FormUrlEncodedContent(content))
            {
                query = encodedContent.ReadAsStringAsync().Result;
            }
            result = await _client.GetAsync($"{uri}/?{query}");
        }
        else if (method == HttpMethod.Delete)
        {
            string query;
            using (var encodedContent = new FormUrlEncodedContent(content))
            {
                query = encodedContent.ReadAsStringAsync().Result;
            }
            result = await _client.DeleteAsync($"{uri}/?{query}");
        }
        else
        {
            var httpContent = new HttpRequestMessage(method, uri);
            if (content != null)
            {
                httpContent.Content = new FormUrlEncodedContent(content.ToArray());
            }
            result = await _client.SendAsync(httpContent);
        }
        string resultContent = await result.Content.ReadAsStringAsync();
        if (result.StatusCode == System.Net.HttpStatusCode.RequestUriTooLong)
        {
            throw new ArgumentException(ErrorMessages.UriTooLong);
        }
        if (!result.IsSuccessStatusCode)
        {
            throw new Exception(resultContent); 
        }
        return JsonConvert.DeserializeObject<T>(resultContent);
    });
}
\end{lstlisting}

\section{Testing}
La maggior parte dei servizi è stata testata tramite il tool XUnit; principalmente sono stati eseguiti test di validazione, data la natura della libreria, appoggiandosi 
ad una fixture comune a tutti i test per l'inizializzazione dei servizi.

è stato usato fin da subito il servizio online di testing automatico Travis, tramite il quale è stato possibile validare ogni rilascio facendo 
partire automaticamente i test con l'evento di push di Git. 

\section{Documentazione}
Infine le classi e i metodi più rilevanti sono stati commentati tramite i tag XML specificati nella documentazione Microsoft e la documentazione in formato 
HTML è stata generata automaticamente tramite il tool Wyam.

\section{Nuget}
La \code{.dll} generata dalla compilazione della libreria è stata infine documentata sotto il profilo delle dipendenze, generando il file \code{.nuspec}, ed inclusa nel file 
\code{.nupkg} tramite l'apposito tool fornito da Nuget. Il tutto è stato caricato sul portale online Nuget ed è ora possibile includerlo in un progetto tramite il comando cli:

\begin{lstlisting}[style=TexStyle, caption=Comando cli per includere il client in un progetto]    
$ dotnet add package SpazioDati.Dandelion
\end{lstlisting}

\section{Demo}
Per provare il client basta creare un programma cli di test:

\begin{lstlisting}[style=TexStyle, caption=Creazione di un progetto demo]    
    $ mkdir Demo && cd Demo
    $ dotnet new console
    $ dotnet add package SpazioDati.Dandelion
    $ dotnet restore
\end{lstlisting}

Nel file \code{Program.cs} sostituire il seguente codice e lanciare il programma:
\begin{lstlisting}[style=CSharpStyle, caption=Codice per chiamare Entity Extraction API]    
    using System;
    using System.Net.Http;
    using Newtonsoft.Json;
    using SpazioDati.Dandelion.Business;
    using SpazioDati.Dandelion.Domain.Models;
    
    namespace Demo
    {
        class Program
        {
            static void Main(string[] args)
            {
                var token = "<your-token>";
                var text = "the quick brown fox jumps over the lazy dog";
                var entities = DandelionUtils.GetEntitiesAsync(new EntityExtractionParameters{Token = token, Text = text, HttpMethod = HttpMethod.Post});
                Console.WriteLine(JsonConvert.SerializeObject(entities));
            }
        }
    }
\end{lstlisting}   

\begin{lstlisting}[style=TexStyle, caption=Compilare e lanciare il programma]    
    $ dotnet run
\end{lstlisting}
