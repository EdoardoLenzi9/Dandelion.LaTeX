\chapter{Conclusioni}
Personalmente ritengo che l'esperienza di tesi sia stata molto positiva in quanto ho avuto la possibilità di entrare in contatto con diverse realtà lavorative fuori dal comune; 
sotto il profilo tecnico questo progetto mi ha portato ad apprendere molte nuove tecnologie che non conoscevo e a rafforzare conoscenze pregresse.

La prima parte, se pur non particolarmente complessa, è stata interessante perchè riguardava l'intero ciclo di vita di una libreria, dall'implementazione alla pubblicazione;
mi ha dato modo di rafforzare le nozioni di programmazione in C$\#$ e di seguire, per la prima volta, il deploy di una libreria su Nuget. 
La libreria è funzionante e ad oggi conta un centinaio di download, risultato sicuramente positivo.

La seconda parte si è rivelata, alla fine, un'analisi fine a se stessa, non si sono ottenuti risultati utili per poter ottimizzare l'algoritmo esistente;
nonostante ciò la ritengo personalmente un'esperienza positiva dato che mi ha permesso di provare svariate nuove tecnologie molto utili, prima fra tutte Docker. 

Mi ha colpito molto l'elevata complessità insita nel problema della valutazione prestazionale di un processo; 
infatti nel corso dell'analisi sono stati provati svariati javaagent, per monitorare il processo, che regolarmente risultavano inattendibili, 
in parte per l'interazione con altri processi e in parte per il ridotto scope di visibilità del software di monitoraggio stesso 
(senza contare la presenza di svariati fattori aleatori come l'azione del Garbage Collector).

Altra cosa che mi ha sorpreso molto è stato l'incredibile calo prestazionale della funzione hash all'aumentare delle dimensioni del dominio, tanto che la computazione dell'hash stesso 
richiedeva più tempo della ricerca binaria.

Penso sia comunque possibile trovare un'implementazione migliore dell'attuale; sarebbe un interessante spunto di ricerca ulteriore valutare se sia possibile creare una 
funzione hash ad hoc per questo problema; sarebbe anche interessante valutare implementazioni basate su strutture dati salvate su disco anzichè in memoria 
(database, indici di Lucene~\cite{lucene}, ecc.), anche se sicuramente i tempi di risposta aumenterebbero di qualche ordine di grandezza per il solo accesso al disco. 

La terza parte mi ha permesso di conoscere il mondo di Wikidata e del semantic web più in generale, dandomi l'opportunità di contribuire, per la prima volta, 
ad un progetto open source così rilevante.

Si è ottenuto il risultato sperato, in quanto lo script funziona correttamente, tuttavia sarebbe interessante riuscire ad ottimizzare lo script visto che attualmente, 
per computare un dataset da 500.000 righe, impiega qualche ora; 
va da sè che spesso si hanno ritardi, anche di svariati secondi, totalmente indipendenti dallo script (connessione ad internet, tempo di attesa per le query SPARQL 
e per il "refresh" delle URL deprecate, tramite chiamate HTTP); tolti questi punti sicuramente si possono ottimizzare le routine maggiormente usate nello script 
per abbassare i tempi di computazione.

In conclusione voglio ringraziare tutti coloro che mi hanno aiutato e seguito in questo progetto di tesi.