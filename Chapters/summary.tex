\chapter*{Sommario} % senza numerazione
\label{sommario}

Il progetto di tesi è il frutto di un tirocinio presso SpazioDati ed una successiva collaborazione al progetto Wikidata StrepHit. La tesi consiste principalmente in tre parti separate, ogniuna delle quali è
legata al servizio online Dandelion (creato da SpazioDati), che verrà brevemente introdotto nelle prossime pagine. 

Il primo capitolo ha lo scopo di introdurre, al lettore, gli applicativi, le tecnologie ed i servizi usati durante il progetto di tesi, descrivendoli sommariamente; 
nei tre capitoli successivi, invece, si entrerà maggiormente nel dettaglio del progetto descrivendo i tre ambiti fondamentali su cui si è focalizzata la tesi.

La prima parte del progetto consiste nell'implementazione di un client in C$\#$, finalizzato allo scopo di creare una libreria di metodi \lq\lq plug$\&$play\rq\rq\ 
per chiamare automaticamente le RESTful-API del servizio Dandelion. 

Durante lo sviluppo il codice è stato regolarmente caricato online, sulla piattaforma GitHub, per agevolare la periodica revisione del codice effettuata dal team di SpazioDati; 
per rendere lo sviluppo più efficiente si è cercato di seguire i principi del TDD (Test Driven Development), affidandosi al servizio di testing automatico Travis 
per validare automaticamente ogni pull-request su GitHub. 

Una volta completata la libreria e l'annessa documentazione è stata compilata la dll e resa pubblica su Nuget; 
pertanto la libreria è ora facilmente integrabile in qualsiasi progetto C$\#$.

La seconda parte del progetto riguarda l'analisi e il test di un algoritmo presente nel backend di Dandelion al fine di ottimizzarlo. 
Il task nasce dalla necessità pratica di ridurre la memoria necessaria alle macchine di produzione di SpazioDati per eseguire l'algoritmo senza subire cali prestazionali.
Partendo dall'implementazione attualmente in uso si sono studiate delle implementazioni alternative per massimizzare la velocità dell'algoritmo e minimizzare 
la memoria occupata dalle strutture dati di appoggio dell'algoritmo. 

L'ultima parte gravita attorno al progetto StrepHit di Wikidata, un progetto nato un anno fa in FBK ed approvato dalla comunity di Wikidata. 
Il progetto nasce per arricchire i database di Wikidata con riferimenti a fonti esterne (tendenzialmente altri siti web) in modo da dare all'utente, fruitore dell'enciclopedia, informazioni sempre più corrette,
validate anche da un sito esterno e non più solo dalla comunity di Wikipedia/Wikidata.

Il progetto è fortemente legato a Dandelion perchè nelle logiche interne dello script di StrepHit si fa largo uso dei servizi offerti da Dandelion per l'analisi testuale delle pagine dei siti esterni.
L'ultima attività consiste nell'implementazione di uno script in python (versione 2.7) utile ad arricchire un dataset di quickstatements aggiungendo riferimenti a proprietà di Wikidata.
Per fare ciò è stato necessario studiare il linguaggio python ed il funzionamento generale di Wikidata per poi realizzare uno script capace di interfacciarsi con l'endpoint SPARQL di Wikidata.
