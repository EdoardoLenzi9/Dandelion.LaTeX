\chapter*{Sommario} % senza numerazione
\label{sommario}

\addcontentsline{toc}{chapter}{Sommario} % da aggiungere comunque all'indice

Il progetto di tesi consiste in una prima parte riguardante l'implementazione di un client in C$\#$, finalizzato allo scopo di includere 
in una libreria metodi plug$\&$play per chiamare automaticamente le RESTful-API del servizio online Dandelion. 

Durante lo sviluppo il codice \'e stato regolarmente caricato su GitHub per una periodica revisione da parte del team di SpazioDati; 
per rendere lo sviluppo pi\'u efficiente si \'e cercato di seguire i principi del TDD, affidandosi al servizio online Travian 
per validare automaticamente ogni rilascio. 

Una volta completata la libreria, \'e stata stilata la documentazione (esportata in html tramite il tool Wyam) e caricata la dll su Nuget; 
pertanto la libreria \'e ora facilmente integrabile in qualsiasi progetto C$\#$.

La seconda parte del progetto di tesi riguarda sempre il servizio Dandelion di SpazioDati ma, questa volta, una parte del codice back-end.
Partendo dall'implementazione attualmente in uso si sono studiate delle implementazioni alternative per massimizzare la velocit\'a dell'algoritmo e minimizzare 
la memoria occupata dalle strutture dati di appoggio dell'algoritmo. 

...

Al termine dell'esperienza si \'e concluso che ...
%TODO conclusioni, metodologie e tecniche usate, i dati elaborati 
%  contesto e motivazioni 
%  breve riassunto del problema affrontato
%  tecniche utilizzate e/o sviluppate
%  risultati raggiunti, sottolineando il contributo personale del laureando/a




