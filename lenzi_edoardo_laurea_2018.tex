% Template per Elaborato di Laurea
% DISI - Dipartimento di Ingegneria e Scienza dell’Informazione

% formato FRONTE RETRO
\documentclass[epsfig,a4paper,11pt,titlepage,twoside,openany]{book}
\usepackage{epsfig}
\usepackage{plain}
\usepackage{setspace}
\usepackage[paperheight=29.7cm,paperwidth=21cm,outer=1.5cm,inner=2.5cm,top=2cm,bottom=2cm]{geometry} % per definizione layout
\usepackage{titlesec} % per formato custom dei titoli dei capitoli
%\usepackage[italian]{babel}
\usepackage[utf8x]{inputenc} % per Linux (richiede il pacchetto unicode); lettere accentate
\usepackage{listings}
\usepackage{color}
\usepackage{amsmath}
\usepackage{amsfonts}
\usepackage{amssymb}
\usepackage{hyperref} 
\usepackage{xcolor}
\usepackage{color,soul}
\usepackage[italian]{babel} 
\usepackage[bottom]{footmisc}

\usepackage{graphicx} % Required for including images
\usepackage[font=small,labelfont=bf]{caption} 
\definecolor{customLightGray}{RGB}{239, 240, 241}
\DeclareRobustCommand{\code}[1]{{\sethlcolor{customLightGray}\hl{#1}}}
\DeclareRobustCommand{\qt}[1]{{\lq\lq\textit{#1}\rq\rq }}
\DeclareRobustCommand{\u}[0]{{\textunderscore}}

\definecolor{dkgreen}{rgb}{0,0.6,0}
\definecolor{gray}{rgb}{0.5,0.5,0.5}
\definecolor{mauve}{rgb}{0.58,0,0.82}

\lstdefinestyle{CSharpStyle}{
  frame=tb,
  language=[Sharp]C,
  aboveskip=3mm,
  belowskip=3mm,
  showstringspaces=false,
  columns=flexible,
  basicstyle={\small\ttfamily},
  numbers=left,
  numberstyle=\tiny\color{gray},
  keywordstyle=\color{blue},
  commentstyle=\color{dkgreen},
  stringstyle=\color{mauve},
  breaklines=true,
  breakatwhitespace=true,
  tabsize=3
}

\lstdefinestyle{JavaStyle}{
  frame=tb,
  language=Java,
  aboveskip=3mm,
  belowskip=3mm,
  showstringspaces=false,
  columns=flexible,
  basicstyle={\small\ttfamily},
  numbers=left,
  numberstyle=\tiny\color{gray},
  keywordstyle=\color{blue},
  commentstyle=\color{dkgreen},
  stringstyle=\color{mauve},
  breaklines=true,
  breakatwhitespace=true,
  tabsize=3
}

\lstdefinestyle{TeXStyle}{
  frame=tb,
  language=TeX,
  aboveskip=3mm,
  belowskip=3mm,
  showstringspaces=false,
  columns=flexible,
  basicstyle={\small\ttfamily},
  numbers=left,
  numberstyle=\tiny\color{gray},
  keywordstyle=\color{blue},
  commentstyle=\color{dkgreen},
  stringstyle=\color{mauve},
  breaklines=true,
  breakatwhitespace=true,
  tabsize=3
}

\lstdefinestyle{YmlStyle}{
  frame=tb,
  language=TeX,
  aboveskip=3mm,
  belowskip=3mm,
  showstringspaces=false,
  columns=flexible,
  basicstyle={\small\ttfamily},
  numbers=left,
  numberstyle=\tiny\color{gray},
  keywordstyle=\color{blue},
  commentstyle=\color{dkgreen},
  stringstyle=\color{mauve},
  breaklines=true,
  breakatwhitespace=true,
  tabsize=3
}

\lstdefinestyle{QuickstatementsStyle}{
  frame=tb,
  language=TeX,
  aboveskip=3mm,
  belowskip=3mm,
  showstringspaces=false,
  columns=flexible,
  basicstyle={\small\ttfamily},
  numbers=left,
  numberstyle=\tiny\color{gray},
  keywordstyle=\color{blue},
  commentstyle=\color{dkgreen},
  stringstyle=\color{mauve},
  breaklines=true,
  breakatwhitespace=true,
  tabsize=3
}

\lstdefinestyle{SPARQLStyle}{
  frame=tb,
  language=TeX,
  aboveskip=3mm,
  belowskip=3mm,
  showstringspaces=false,
  columns=flexible,
  basicstyle={\small\ttfamily},
  numbers=left,
  numberstyle=\tiny\color{gray},
  keywordstyle=\color{blue},
  commentstyle=\color{dkgreen},
  stringstyle=\color{mauve},
  breaklines=true,
  breakatwhitespace=true,
  tabsize=3
}

\lstdefinestyle{jsonStyle}{
  frame=tb,
  language=TeX,
  aboveskip=3mm,
  belowskip=3mm,
  showstringspaces=false,
  columns=flexible,
  basicstyle={\small\ttfamily},
  numbers=left,
  numberstyle=\tiny\color{gray},
  keywordstyle=\color{blue},
  commentstyle=\color{dkgreen},
  stringstyle=\color{mauve},
  breaklines=true,
  breakatwhitespace=true,
  tabsize=3
}

\singlespacing

\begin{document}
  \pagenumbering{gobble} 
  \input{Chapters/frontispiece.tex}
  \clearpage
  \begin{center}
  {\bf \Huge Ringraziamenti}
\end{center}

\vspace{4cm}

\emph{
Prima di addentrarmi nella descrizione del progetto di tesi voglio ricordare coloro che mi sono stati vicini.
}

\emph{
Ringrazio la mia famiglia per essermi sempre stata accanto ed avermi supportato (e sopportato) in tutti questi anni, permettendomi di studiare fino a laurearmi; ringrazio in particolare mio fratello Massimo e mia sorella Maria Vittoria che più di chiunque altro mi hanno aiutato nello studio.
}

\emph{
Ringrazio i miei zii Chiara e Christian, persone uniche e speciali che da sempre mi sono vicine e mi sostengono. 
}

\emph{
Ringrazio i miei amici, in particolare Michele, dispensatore dei migliori consigli e delle peggiori critiche, senza il quale non avrei nemmeno scelto questa via.
}

\emph{
Ringrazio i miei colleghi di lavoro che mi hanno insegnato moltissimo, facendomi appassionare, di giorno in giorno, sempre più al mio lavoro. 
}

\emph{
Devo, infine, un ringraziamento particolare ad Alberto Montresor, Davide Setti, Alessio Guerrieri, Ugo Scaiella e Marco Fossati che mi hanno seguito ed aiutato moltissimo in questo percorso di tesi.
}

\emph{
A tutti loro va la mia riconoscenza ed i miei più sentiti ringraziamenti.
}

  \clearpage
  \pagestyle{plain} % nessuna intestazione e pie pagina con numero al centro
  % inizio numerazione pagine in numeri arabi
  \mainmatter

  %% Nel conteggio delle facciate (massimo 30) sono incluse: indice, sommario, capitoli
  %% Dal conteggio delle facciate sono escluse: frontespizio, ringraziamenti, allegati

    \tableofcontents % indice
    \clearpage
             
    % gruppo per definizone di successione capitoli senza interruzione di pagina
    \begingroup
      % nessuna interruzione di pagina tra capitoli
      % ridefinizione dei comandi di clear page
      \renewcommand{\cleardoublepage}{} 
      \renewcommand{\clearpage}{} 
      % redefinizione del formato del titolo del capitolo
      % da <Capitolo X> a <X Titolo capitolo>
      
      \titleformat{\chapter}{\normalfont\Huge\bfseries}{\thechapter}{1em}{}
      \titlespacing*{\chapter}{0pt}{0.59in}{0.02in}
      \titlespacing*{\section}{0pt}{0.20in}{0.02in}
      \titlespacing*{\subsection}{0pt}{0.10in}{0.02in}
      
      \chapter*{Sommario} % senza numerazione
\label{sommario}

Il progetto di tesi è il frutto di un tirocinio presso SpazioDati\footnote{
    \textbf{SpazioDati} è un'azienda con una sede operativa a Trento ed una a Firenze che si occupa principalmente di tecnologie Big Data e Semantic Web.
    Come riportato sul loro sito \href{https://spaziodati.eu/it/}{spaziodati.eu} stanno costruendo un knowledge-graph di alta qualità, accessibile attraverso delle semplici API 
    su \href{https://dandelion.eu/}{dandelion.eu}.
}
ed una successiva collaborazione al progetto Wikidata StrepHit\footnote{
    \textbf{StrepHit} è un progetto di \href{https://www.wikidata.org/wiki/Wikidata:Main_Page}{Wikidata} nato l'anno scorso con il fine di implementare uno \href{https://www.mediawiki.org/wiki/StrepHit}{script} 
    capace di analizzare testi e tradurli in Wikidata statements (o \href{https://www.wikidata.org/wiki/Help:QuickStatements}{QuickStatements}). 
}. 

La tesi si divide principalmente in tre parti separate, ogniuna delle quali legata al servizio online Dandelion\footnote{
    \textbf{Dandelion} è un servizio online fornito da SpazioDati che mette a disposizione dell'utente servizi di analisi semantica testuale.
}, che verrà brevemente introdotto nelle prossime pagine. 

Il primo capitolo ha lo scopo di introdurre, al lettore, gli applicativi, le tecnologie ed i servizi usati durante il progetto di tesi, descrivendoli sommariamente; 
nei tre capitoli successivi, invece, si entrerà maggiormente nel dettaglio del progetto descrivendo i tre ambiti fondamentali su cui si è focalizzata la tesi.

La prima parte del progetto consiste nell'implementazione di un client in C$\#$, finalizzato allo scopo di creare una libreria di metodi \lq\lq plug$\&$play\rq\rq\ 
per chiamare automaticamente le RESTful-API del servizio Dandelion. 

Durante lo sviluppo il codice è stato regolarmente caricato online, sulla piattaforma \href{https://github.com/}{GitHub}, per agevolare la periodica revisione del codice effettuata dal team di SpazioDati; 
per rendere lo sviluppo più efficiente si è cercato di seguire i principi del TDD (Test Driven Development), affidandosi al servizio di testing automatico \href{https://www.travis-ci.com/}{Travis} 
per validare automaticamente ogni pull-request su GitHub. 

Una volta completata la libreria e l'annessa documentazione è stata compilata la dll e resa pubblica su \href{https://www.nuget.org/}{Nuget}; 
pertanto la libreria è ora facilmente integrabile in qualsiasi progetto C$\#$.

La seconda parte del progetto riguarda l'analisi e il test di un algoritmo presente nel backend di Dandelion al fine di ottimizzarlo. 
Il task nasce dalla necessità pratica di ridurre la memoria necessaria alle macchine di produzione di SpazioDati per eseguire l'algoritmo senza subire cali prestazionali.
Partendo dall'implementazione attualmente in uso si sono studiate delle implementazioni alternative per massimizzare la velocità dell'algoritmo e minimizzare 
la memoria occupata dalle strutture dati di appoggio dell'algoritmo. 

L'ultima parte gravita attorno al progetto StrepHit di Wikidata, un progetto nato un anno fa in \href{https://www.fbk.eu/it/}{FBK} ed approvato dalla comunity di Wikidata. 
Il progetto nasce per arricchire i database di Wikidata con riferimenti a fonti esterne (tendenzialmente altri siti web) in modo da dare all'utente, fruitore dell'enciclopedia, informazioni sempre più corrette,
validate anche da un siti esterni e non più solo dalla comunity di Wikipedia/Wikidata.

Il progetto è fortemente legato a Dandelion perchè nelle logiche interne dello script di StrepHit si fa largo uso dei servizi offerti da Dandelion per l'analisi testuale delle pagine dei siti esterni.
L'ultima attività consiste nell'implementazione di uno script in Python (versione 2.7) utile ad arricchire un dataset di quickstatements aggiungendo riferimenti a proprietà di Wikidata.
Per fare ciò è stato necessario studiare il linguaggio Python ed il funzionamento generale di Wikidata per poi realizzare uno script capace di interfacciarsi con l'endpoint SPARQL\footnote{
    Wikidata si basa su un knowledge base \href{https://www.w3.org/RDF/}{RDF} e mette a disposizione un end-point per eseguire query su quest'ultimo; \href{https://www.w3.org/TR/rdf-sparql-query/}{SPARQL} 
    è il linguaggio di query definito da \href{https://www.w3.org/}{W3C} per il framework RDF, adottato da questo end-point.
} di Wikidata.
 %massimo 3 pagine 
      \addcontentsline{toc}{chapter}{Sommario} % da aggiungere comunque all'indice
      %%   contesto e motivazioni, breve riassunto del problema affrontato, tecniche utilizzate e/o sviluppate, risultati raggiunti
      \newpage
      \chapter{Introduzione}
\section{Dandelion}

Dandelion è un servizio online fornito da SpazioDati che mette a disposizione dell'utente servizi di analisi semantica testuale; 
grazie ad esso è possibile, dato un testo, estrarne le entità semantiche principali (\textit{Entity Extraction}), trovare la lingua in cui è stato 
scritto (\textit{Language Detection}), classificarlo secondo modelli definiti dall'utente stesso (\textit{Text Classification}) e analizzarne la semantica 
per capire i sentimenti che l'autore ci vuole trasmettere (\textit{Sentiment Analysis}). 

Dandelion ha anche altre due RESTful-API che permettono di confrontare due testi generando un indice di similitudine fra i due (\textit{Text Similarity}) e
un motore di ricerca di entità di Wikipedia (\textit{Wikisearch}), nel caso si voglia trovare il titolo di un contenuto senza conoscerlo a priori.

L'enciclopedia alla base di Dandelion è Wikipedia, anche se a volte fra i due si colloca come mediatore dbpedia, un progetto italiano per l'estrazione di 
informazioni semi-strutturate da Wikipedia. 

Per poter usare gli end-point di Dandelion basta registrarsi sul portale dedicato e generare un token che andrà inserito come query parameter nelle chiamate https alle API;
la documentazione delle API è disponibile all'indirizzo: https://dandelion.eu/docs/.

[TODO descrizione più dettagliata di Dandelion]
%\subsection{Entity Extraction}
%\subsection{Text Similarity}
%\subsection{Text Classification}
%\subsection{Language Detection}
%\subsection{Sentiment Analysis}
%\subsection{Wikisearch}

\section{GitHub e Travis}
Per il versioning del codice si è scelto di usare il software git, appoggiandosi alla piattaforma GitHub per creare repository pubbliche. 

Le repository riguardanti il progetto di tesi sono:

\begin{itemize}
    \item Repository contenente il testo della tesi scritto in \LaTeX\ (\url{https://github.com/EdoardoLenzi9/Dandelion.LaTeX}) \\
    \item Repository contenente lo script in Python per il progetto StrepHit (\url{https://github.com/EdoardoLenzi9/Wikipedia.StrepHit}) \\
    \item Repository contenente il codice Java con le varie implementazioni dell'algoritmo per il calcolo della relatedness (la seconda parte del progetto) (\url{https://github.com/EdoardoLenzi9/Dandelion.Relatedness}) \\
    \item Repository contenente il codice sorgente del client C$\#$ (\url{https://github.com/EdoardoLenzi9/SpazioDati.Dandelion-eu})\\
\end{itemize} 

Per quanto riguarda il testing automatico è stata scelta la piattaforma Travis che permette di agganciare una repository GitHub su cui eseguire automaticamente i test 
ad ogni commit. 

Per fare ciò è necessario inserire nella repository un file di configurazione chiamato .travis.yml con i comandi necessari ad eseguire i test. 

Per esempio nel caso del client C$\#$ il file yml contiene le seguenti istruzioni:

\begin{lstlisting}[style=YmlStyle]
    language: csharp
    dist: trusty
    mono: none
    dotnet: 2.0.3

    install:
    - dotnet restore

    script:
    - dotnet build
    - dotnet test SpazioDati.Dandelion.Test/SpazioDati.Dandelion.Test.csproj
\end{lstlisting}

Mentre per lo script in python la sintassi è la seguente:
\begin{lstlisting}[style=YmlStyle]
    language: python
    python: 
      - "2.7"
    # command to install dependencies
    install:
      - pip install -r requirements.txt
    # command to run tests
    script:
      - python test.py
\end{lstlisting}

Come si nota dalle righe sopra, basta specificare il linguaggio, la versione ed i comandi (Travis esegue i test in ambiente Linux, quindi i comandi devono essere eseguibili in una bash-shell Linux) 
per installare e lanciare i test.

\section{Docker}
Per la seconda parte del progetto è stato usato Docker per creare un ambiente "chiuso", ad hoc, dove poter testare al meglio gli algoritmi senza doversi preoccupare di fattori esterni che potrebbero falsare 
il risultato del test. 

Docker permette di creare un'immagine a partire da un file di testo chiamato \lq\lq Dockerfile\rq\rq\ il quale contiene tutte le informazioni per eseguire la build dell'immagine; in pratica 
il Docker file è una sorta di descrittore dello stato di una macchina virtuale che genera una macchina virtuale persistente chiamata immagine. L'immagine può essere istanziata, divenendo un container 
il quale però non è più persistente. Una delle caratteristiche più utili di Docker è la semplicità con cui è possibile interfacciare container diversi condividendo aree di memoria sul disco e porte TCP.

Per esempio l'idea di base per monitorare le prestazioni dell'algoritmo e l'allocazione di memoria nel Docker è quella di eseguire il file jar compilato, dentro un container, con un particolare javaagent chiamato 
\href{https://github.com/prometheus/jmx_exporter}{jmx-exporter}; l'exporter si occupa di monitorare l'applicativo e di esportare in tempo reale delle metrics che vengono esposte in un server locale su una porta a scelta.
Parallelamente al primo container vengono istanziate altre due immagini, scaricate da \href{https://hub.docker.com/}{hub.docker}, chiamate prom/prometheus e grafana/grafana le quali permettono di filtrare le metrics 
generate dall'exporter e di visualizzarle direttamente su dei grafici. 

Per il setup dell'ambiente è stato creata un'immagine a partire dal seguente Dockerfile:

\begin{lstlisting}[style=YmlStyle]
    FROM openjdk:latest
    COPY . /usr/src/dandelion
    WORKDIR /usr/src/dandelion
    CMD ["java", "-Xmx6144m", "-javaagent:./lib/jmx_prometheus_javaagent-0.3.1.jar=8080:./lib/configs.yaml", "-jar", "/usr/src/dandelion/dist/Dandelion.jar"]
\end{lstlisting}

Sostanzialmente l'immagine viene edificata sopra un'altra immagine chiamata \href{https://hub.docker.com/_/openjdk/}{openjdk}, scaricata da \href{https://hub.docker.com/}{hub.docker}, contenente 
un'implementazione open source di Java SE (Java Platform, Standard Edition) a partire dalla versione 7. 
La direttiva COPY serve per copiare, al momento della build dell'immagine, tutto il contenuto della cartella contenente dentro l'ambiente virtuale alla path specificata (/usr/src/dandelion); mentre WORKDIR specifica 
la path della \textit{working directory} da cui partirà il container docker. 
Infine la direttiva CMD lancia il .jar generato dalla compilazione con il tool ant e setta come javaagent jmx-exporter (l'opzione -Xmx6144m serve solamente a specificare la massima allocazione di memoria heap, fino ad un massimo di 6GB).   

\section{Prometheus e Grafana}
[TODO]

\section{Wikidata e StrepHit}
[TODO]

\subsection{SPARQL}
[TODO]

      \newpage
      \addcontentsline{toc}{chapter}{Client C$\#$}

\chapter{Client C$\#$}
%%%%%%%%%%%%%%%%%%%%%%%%%%%%%%%%%%%%%%%%%%%%%%%%%%%%%%%%%%%%%%%%%%%%%%%%%%%%%%%%%%%%%%%%%%%%%%%%%%%%%%%%%%%%%%%%%%%%%%%%
\section{Struttura Generale}

Per l'implementazione si è cercato di seguire le best practices, dettate dalle linee guida Microsoft, per la stesura del codice.
Sono stati adottati, come pattern di programmazione, dependency injection, MVC (per quanto possibile) e TDD, appoggiandosi a librerie esterne Nuget\footnote{
    \textbf{Nuget} è un package manager per .NET. \href{https://www.nuget.org/}{Sito ufficiale}
} quali 
\href{https://www.nuget.org/packages/newtonsoft.json/}{Newtonsoft.Json} e \href{https://www.nuget.org/packages/SimpleInjector/}{SimpleInjector}. I metodi sono asincroni e le property thread-safe.

La solution è stata partizionata in vari progetti; il progetto Business contiene la business-logic 
(comprendente servizi, metodi estensione, l'implementazione del client e un wrapper del Container di SimpleInjector) e utilizza i modelli di Domain.

In Test sono contenute le classi di testing e le fixture XUnit\footnote{
    \textbf{XUnit} è una libreria \href{https://www.nuget.org/packages/xunit/}{Nuget} che mette a disposizione un tool per eseguire unit test. \href{https://github.com/xunit/xunit}{Repository GitHub}    
}, mentre in Documentation sono presenti i file generati dal tool Wyam\footnote{
    \textbf{Wyam} è un tool che permette l'esportazione della documentazione del codice C$\#$ 
    (sottoforma di \href{https://docs.microsoft.com/en-us/dotnet/csharp/codedoc}{tag xml}) in pagine html. \href{https://wyam.io/}{Sito ufficiale}
} per la documentazione.

%%%%%%%%%%%%%%%%%%%%%%%%%%%%%%%%%%%%%%%%%%%%%%%%%%%%%%%%%%%%%%%%%%%%%%%%%%%%%%%%%%%%%%%%%%%%%%%%%%%%%%%%%%%%%%%%%%%%%%%%
\section{Il Client}

Dalla documentazione ufficale delle API di Dandelion si evince che ogni end-point richiede uno o più testi da analizzare ed una serie di parametri, alcuni obbligatori ed altri 
opzionali; 
pertanto sono stati creati dei modelli, coerenti con tali definizioni, che, passati in argomento a dei servizi, ritornano i DTO delle risposte degli end-point di Dandelion.

Ogni servizio controlla che i parametri inseriti dall'utente rispettino i constraints definiti nella documentazione e costruisce, tramite i metodi \code{ContentBuilder()} e  
\code{UriBuilder()}, un dizionario di parametri e l'URI che identifica l'end-point. 

Infine il servizio chiama il metodo generico \code{CallApiAsync()}, passandogli il content, la URI ed il metodo HTTP; 
nel metodo di chiamata si è scelto di gestire separatamente i metodi HTTP GET e DELETE dato che essi non permettono il passaggio parametri nel boby 
della chiamata. Pertanto è stato necessario inviare il content come query parameter facendone il percent-encoding.

Dandelion ammette sia chiamate GET che POST ai servizi, tuttavia nel caso del metodo GET non è garantito il corretto 
funzionamento del servizio per testi che superano i 2000 caratteri; pertanto nel caso in cui l'utente non specifichi il metodo HTTP, di default verrà
usato il metodo POST.

Se la chiamata è andata a buon fine il JSON di ritorno viene mappato in un DTO specifico, a seconda del servizio scelto, e restitiuito direttamente all'utente.

\begin{lstlisting}[style=CSharpStyle, caption=Metodo generico del client per le chiamate HTTP]
public Task<T> CallApiAsync<T>(string uri, List<KeyValuePair<string, string>> content, HttpMethod method = null)
{
    var result = new HttpResponseMessage();
    if (method == null)
    {
        method = HttpMethod.Post;
    }

    if (_client.BaseAddress == null)
    {
        _client.BaseAddress = new Uri(Localizations.BaseUrl);
    }

    return Task.Run(async () =>
    {
        if (method == HttpMethod.Get)
        {
            string query;
            using (var encodedContent = new FormUrlEncodedContent(content))
            {
                query = encodedContent.ReadAsStringAsync().Result;
            }
            result = await _client.GetAsync($"{uri}/?{query}");
        }
        else if (method == HttpMethod.Delete)
        {
            string query;
            using (var encodedContent = new FormUrlEncodedContent(content))
            {
                query = encodedContent.ReadAsStringAsync().Result;
            }
            result = await _client.DeleteAsync($"{uri}/?{query}");
        }
        else
        {
            var httpContent = new HttpRequestMessage(method, uri);
            if (content != null)
            {
                httpContent.Content = new FormUrlEncodedContent(content.ToArray());
            }
            result = await _client.SendAsync(httpContent);
        }
        string resultContent = await result.Content.ReadAsStringAsync();
        if (result.StatusCode == System.Net.HttpStatusCode.RequestUriTooLong)
        {
            throw new ArgumentException(ErrorMessages.UriTooLong);
        }
        if (!result.IsSuccessStatusCode)
        {
            throw new Exception(resultContent); 
        }
        return JsonConvert.DeserializeObject<T>(resultContent);
    });
}
\end{lstlisting}

\section{Testing}
La maggiorparte dei servizi è stata testata tramite il tool XUnit; principalmente sono stati eseguiti test di validazione, data la natura della libreria, appoggiandosi 
ad una fixture comune a tutti i test per l'inizializzazione dei servizi.

è stato usato fin da subito il servizio online di testing automatico Travis, tramite il quale è stato possibile validare ogni rilascio facendo 
partire automaticamente i test con l'evento di push di Git. 

\section{Documentazione}
Infine le classi e i metodi più rilevanti sono stati commentati tramite i tag XML specificati nella documentazione Microsoft e la documentazione in formato 
HTML è stata generata automaticamente tramite il tool Wyam.

\section{Nuget}
La dll generata dalla compilazione della libreria è stata infine documentata sotto il profilo delle dipendenze, generando il file .nuspec, ed inclusa nel file .nupkg
tramite l'apposito tool fornito da Nuget. Il tutto è stato caricato sul portale online Nuget ed è ora possibile includerlo in un progetto tramite il comando cli:

\begin{lstlisting}[style=TexStyle]
$ dotnet add package SpazioDati.Dandelion
\end{lstlisting}
      \newpage
      \chapter{Calcolo della Relatedness}

La seconda parte del progetto riguarda l'analisi di due metodi critici (\code{readDump()} e \code{rel(int a, int b)}) che servono a 
calcolare, dati gli identificativi di due entità semantiche, il loro valore di correlazione (relatedness). 

Per correlazione fra due entità si intende, in questo contesto, un indice numerico che assume valori nell'intervallo [0,255], 
riscalato, per comodità, in una percentuale codificata in variabili di tipo \code{float}. 

Si ha a disposizione un dump dove sono salvati tutti i valori di correlazione (sopra una certa soglia minima) per ogni coppia di entità. 

Il calcolo della relatedness viene effettuato a monte,
dai gestori del DataBase che rilasciano il dump; la relatedness è fondamentale per l'algoritmo di Dandelion, basti solo pensare alla potenzialità 
di poter verificare la correlazione fra le varie parole estratte da una frase, controllando quindi anche il contesto oltre alla singola parola.

Il metodo \code{readDump()} serve a caricare in memoria i valori del dump 
sottoforma di \lq\lq matrice\rq\rq\ mentre il metodo \code{rel(int a, int b)} serve a fare una ricerca nella struttura dati per poi ritornare il valore di 
correlazione fra le entità con identificativi \code{a} e \code{b}.

Questa scelta implementativa, attualmente adottata, è sicuramente molto efficiente in termini di prestazioni ma anche molto onerosa in termini di memoria; 
pertanto il fine dell'analisi sarebbe quello di studiare/testare implementazioni alternative per ottimizzare le prestazioni e/o diminuire lo spreco di memoria.  

%%%%%%%%%%%%%%%%%%%%%%%%%%%%%%%%%%%%%%%%%%%%%%%%%%%%%%%%%%%%%%%%%%%%%%%%%%%%%%%%%%%%%%%%%%%%%%%%%%%%%%%%%%%%%%%%%%%%%%%%
\section{Implementazione Iniziale}

%%%%%%%%%%%%%%%%%%%%%%%%%%%%%%%%%%%%%%%%%%%%%%%%%%%%%%%%%%%%%%%%%%%%%%%%%%%%%%%%%%%%%%%%%%%%%%%%%%%%%%%%%%%%%%%%%%%%%%%%
\subsection{Il Dump}
\begin{lstlisting}[style=TeXStyle, caption=Estratto del dump]
53676192
4922289
0.01
2
1.0
53676158 7 null
53676164 5 NnAwQT/ZvTVJ05MeSfcdI0rMrg1LBmUH
53676016 13 RDsFRUkyASRLFf3T
53675811 16 null
...
\end{lstlisting}

Analizzando la parte iniziale di un dump si nota che sulle prime cinque righe ci sono valori di inizializzazione mentre dalla sesta riga in poi troviamo i valori della matrice.

Il primo numero intero, denominato \code{max$\_$id}, indica il limite massimo che gli id delle entità possono assumere; 
dato che gli id delle entità non sono necessariamente sequenziali 
(fra due di essi potrei avere dei \lq\lq buchi\rq\rq, degli intervalli in cui gli identificativi non sono associati ad alcuna entità) 
esiste una funzione \code{map} che mappa (\lq\lq compatta\rq\rq) gli id delle entità su altri id univoci e sequenziali. 

Il secondo intero, denominato \code{nodesSize} è il valore massimo che la funzione \code{map} può assumere (limita il codominio della funzione).

\begin{equation}\begin{split}
    map():\ \mathbb{N} & \rightarrow \mathbb{N}\\
            [0,max\_id] & \rightarrow [0, nodeSize],\ max\_id \geq nodeSize\\
\end{split}\end{equation}

Continuando a leggere il dump troviamo la relatedness minima considerata \code{minRel} (sotto la quale i valori di correlazione non vengono salvati nel dump) 
e altri valori di configurazione quali \code{minIntersection} e \code{threshold} che tuttavia non ci interessano particolarmente.

La funzione \code{map} in un certo senso mappa \lq\lq al contrario\rq\rq\ gli id; infatti l'id massimo verrà mappato su 0, il penultimo id verrà mappato su 1 e così via.

Nel caso del dump in questione abbiamo:

\begin{lstlisting}[style=TeXStyle, caption=Esempio di funzione map e postingList]
MaxId = 53676192
NodeSize = 4922289

   ID		        MAP(ID)    POSTINGLIST
...
53676158    ->      7           null
53676159    ->      6           FW/I...
53676164    ->      5           NnAw...
53676174    ->      4           null
53676176    ->      3           null
53676177    ->      2           null
53676180    ->      1           null
53676190    ->      0           null
\end{lstlisting}

Dalla sesta riga del dump in poi inizia la definizione della matrice, ogni riga è composta da tre valori separati da uno spazio. 

Consideriamo una riga qualsiasi:
\begin{lstlisting}[style=TeXStyle, caption=Riga del dump]
53676164 5 NnAwQT/ZvTVJ05MeSfcdI0rMrg1LBmUH
\end{lstlisting}

Il primo valore \code{$a = 53676164$} è l'id di un entità, il secondo valore \code{$b = 5 = map(a)$} è il risultato della funzione map, 
infine il terzo valore \code{$c = NnAwQT/ZvTVJ05MeSfcdI0rMrg1LBmUH$} è una stringa di dimensione variabile oppure \code{null};
quest'ultima stringa, denominata \code{postingList}, è la codifica in \code{Base64} di un array di byte che rappresenta un dizionario chiave valore. 

\begin{center}
    \includegraphics[scale=0.55]{Sources/Img/c02_01.png}
    \captionof{figure}{Encoding del dizionario nella postingList}
\end{center}

In pratica una tupla \code{$<chiave,\ valore>$} è codificata sull'array da una sequenza di quattro byte; quindi partizionando l'array in gruppi da 4 byte 
ottengo, per ogni gruppo, la chiave (è un id \lq\lq mappato\rq\rq ) codificata sui primi tre byte ed il valore (la relatedness) definito sul quarto byte (il byte più a destra).

Questa codifica, pertanto, permette di gestire set di entità con id \lq\lq mappato\rq\rq\ massimo $2^{3 \cdot 8} \simeq 1,7 \cdot 10^7$ ed una scala di valori 
di correlazione appartenenti all'intervallo $[0,\ 255]$.

Bisogna precisare che l'array di byte, denominato \code{postingList}, è ordinato per valore crescente delle chiavi e, dato che \code{rel(a, b) = rel(b, a)},
si è deciso di salvare una sola volta il valore di correlazione imponendo la relazione: $a > b$. 

Le righe del dump, invece, non sono necessariamente ordiante per identificativo dell'entità.

%%%%%%%%%%%%%%%%%%%%%%%%%%%%%%%%%%%%%%%%%%%%%%%%%%%%%%%%%%%%%%%%%%%%%%%%%%%%%%%%%%%%%%%%%%%%%%%%%%%%%%%%%%%%%%%%%%%%%%%%
\subsubsection{Esempio}
Per calcolare \code{$relatedness(a, d)$} bisogna per prima cosa identificare la riga del dump contenente la tupla \code{$(a,\ b,\ c)$}, 
assumendo che valga la relazione \code{$map(a) > map(d)$}. 

Con una ricerca binaria sulla \code{postingList c} bisogna trovare un gruppo di quattro byte in cui i primi tre byte corrispondano a \code{map(d)}
ed il quarto byte sarà il valore di correlazione voluto. 

Ovviamente è possibile che \code{c} sia \code{null} (l'entità con id \code{a} non ha nessuna correlazione con altre entità i cui id \lq\lq mappati\rq\rq\ siano inferiori a \code{b}) oppure 
che nella \code{postingList} non sia presente nessuna chiave corrispondente a \code{map(d)}, in questi casi \code{$relatedness(a,d) = 0$}. 

%%%%%%%%%%%%%%%%%%%%%%%%%%%%%%%%%%%%%%%%%%%%%%%%%%%%%%%%%%%%%%%%%%%%%%%%%%%%%%%%%%%%%%%%%%%%%%%%%%%%%%%%%%%%%%%%%%%%%%%%
\subsection{Il Codice Nativo}

\begin{lstlisting}[style=JavaStyle, caption=Implentazione nativa]
// STRUTTURA DATI
public class RelatednessMatrix {
    public int[] map;
    public byte[][] matrix;
    public float configuredMinRel;
    public int configuredMinIntersection;
    public float threshold;
    public static int EL_SIZE = 4;
}

// CODICE UTILIZZO
public float rel(int a, int b){
    if(a==b) return 1f;

    int nodeA = data.map[a];
    int nodeB = data.map[b];
    if (nodeA < 0 || nodeB < 0) return 0f;

    int key = a>b ? nodeB : nodeA;
    byte[] array = a>b ? data.matrix[nodeA] : data.matrix[nodeB];

    if (array == null) return 0f;

    int start = 0;
    int end = array.length - RelatednessMatrix.EL_SIZE;
    int pos = -1;
    while(pos == -1 && start <= end)
    {
        int idx = ((start+end)/RelatednessMatrix.EL_SIZE)/2;
        int idx_value = ((array[idx*RelatednessMatrix.EL_SIZE] & 0xFF) << 16)
                    + ((array[idx*RelatednessMatrix.EL_SIZE+1] & 0xFF) << 8)
                    + ( array[idx*RelatednessMatrix.EL_SIZE+2] & 0xFF);

        if(idx_value == key) pos = idx;
        else{
            if(key > idx_value)
                start = (idx + 1) * RelatednessMatrix.EL_SIZE;
            else
                end = (idx - 1) * RelatednessMatrix.EL_SIZE;
        }
    }

    if(pos == -1) return 0f;
    else
    {
        byte by = array[pos * RelatednessMatrix.EL_SIZE + 3];
        int byint = by + 128;
        float byteRel = byint / 255f;
        return data.configuredMinRel + byteRel * (1 - data.configuredMinRel);
    }
}

// CODICE LETTURA
public String END_OF_FILE = "" + '\0';

public RelatednessMatrix readDump() throws Exception {
    URL url = getClass().getResource("dump.txt");
    File file = new File(url.getPath());
    BufferedReader fbr = new BufferedReader(new FileReader(file));
    String line = new String();

    try {   
        int max_id = Integer.parseInt(fbr.readLine().trim()); // Trims all leading and trailing whitespace from this string         
        int nodesSize = Integer.parseInt(fbr.readLine().trim());
        float minRel = Float.parseFloat(fbr.readLine().trim());
        int minIntersection = Integer.parseInt(fbr.readLine().trim());
        float threshold = Float.parseFloat(fbr.readLine().trim());         

        RelatednessMatrix data = new RelatednessMatrix();
        data.map = new int[max_id + 1];
        data.matrix = new byte[nodesSize][];
        data.configuredMinRel = minRel;
        data.configuredMinIntersection = minIntersection;
        data.threshold = threshold;

        int idx = 0;

        while ((line = fbr.readLine()) != null) {
            if (line.trim().equals(END_OF_FILE.trim())) {
                break;
            }
            System.out.println(line);
            String[] splittedLine = line.split("\\s+"); // split(" ")
            if (splittedLine.length != 3) {
                throw new Exception("Wrong format relatedness file for the line: " + line);
            }

            int wid = Integer.parseInt(splittedLine[0]);
            int node = Integer.parseInt(splittedLine[1]);
            data.map[wid] = node;

            if (splittedLine[2].equals("null")) {
                data.matrix[node] = null;
            } else {
                byte[] a = Base64.getDecoder().decode(splittedLine[2].toString()); 
                data.matrix[node] = Base64.getDecoder().decode(splittedLine[2].toString());
            }
            idx++;
        }
        
        if (idx != nodesSize) {
            throw new Exception("Wrong format relatedness file the number of line do not match with size of matrix");
        }
        
        return data;
    } catch (Exception e) {
        System.out.println(e.getMessage());
        return null;
    }
}
\end{lstlisting}

Il metodo \code{readDump()} legge riga per riga il dump e restituisce un'istanza di \code{RelatednessMatrix} valorizzata. 
La funzione map viene salvata in un array di interi (\code{data.map}) mentre la matrice con le correlazioni viene salvata in un 
array bidimensionale di byte (\code{data.matrix}).

Il metodo \code{rel(int a, int b)} calcola la funzione \code{map} sugli id \code{a} e \code{b}, dopodichè fa una binary search sulla \code{postingList} 
ottenuta da \code{matrix[map(a)]} (assumendo $map(a) > map(b)$). 

Se la ricerca binaria ha successo il valore di relatedness ottenuto va riscalato su una scala di valori $[0, 255]$, 
convertito in float e riscalato nuovamente in base al valore di relatedness minima considerata (per escludere i valori al di sotto della relatedness minima,
che non verranno mai usati).  

\begin{lstlisting}[style=JavaStyle, caption=Conversione della relatedness da byte a float]
byte by = array[pos * RelatednessMatrix.EL_SIZE + 3]; //considero il quarto byte 
int byint = by + 128;   //scalo di 128 valori per ottenere una relatedness in [0, 255]

//cast in float, escludendo i valori sotto la relatedness minima (configuredMinRel)
float byteRel = byint / 255f; 
return data.configuredMinRel + byteRel * (1 - data.configuredMinRel);
\end{lstlisting}

%%%%%%%%%%%%%%%%%%%%%%%%%%%%%%%%%%%%%%%%%%%%%%%%%%%%%%%%%%%%%%%%%%%%%%%%%%%%%%%%%%%%%%%%%%%%%%%%%%%%%%%%%%%%%%%%%%%%%%%%
\subsection{Osservazioni}
Considerato che il metodo \code{readDump()} viene chiamato una sola volta per allocare in memoria la struttura dati nel server, 
non ci sono particolari vincoli temporali per la sua esecuzione; per quanto riguarda il calcolo della relatedness invece 
l'algoritmo dev'essere il più performante possibile.

Il codice nativo per il calcolo della relatedness costa $O(lg_2(n))$ (con $n=nodeSize$) per la ricerca binaria e O(1) per quanto riguarda le 
istruzioni precedenti e successive a quest'ultima.

Tolta una piccola modifica al codice che portebbe ad evitare tre moltiplicazioni per ciclo nella binary search (non è necessario dividere e poi moltiplicare per 
\code{RelatednessMatrix.EL$\_$SIZE}), non sembra si possano migliorare di molto le prestazioni dell'algoritmo senza cambiare struttura dati.

Una via percorribile per non \lq\lq pagare\rq\rq\ $O(lg_2(n))$ per la binary search potrebbe essere quella di allocare direttamente in memoria la matrice 
completa (in questo caso avrei un algoritmo di tempo costante O(1)). Questa soluzione però andrebbe a peggiorare drasticamente lo spreco di memoria.
Infatti nell'implementazione attuale la matrice ha dimensione massima $nodeSize \otimes nodeSize$ ma le righe hanno dimensione variabile e 
non occupano quindi sempre $(nodeSize -1) \cdot 4\ Byte$.

%%%%%%%%%%%%%%%%%%%%%%%%%%%%%%%%%%%%%%%%%%%%%%%%%%%%%%%%%%%%%%%%%%%%%%%%%%%%%%%%%%%%%%%%%%%%%%%%%%%%%%%%%%%%%%%%%%%%%%%%
\subsubsection{Stima della Memoria Allocata}
Se consideriamo le prime righe del dump in questione possiamo stimare che mediamente il $57\%$ degli id hanno postingList vuota (null) ed il restante $43\%$ ha un numero 
medio di relatedness per id pari allo $0,0101\%$ di nodeSize.

L'implementazione attuale alloca in memoria per l'array di interi map circa:
\begin{center}
\begin{equation}\begin{split} 
    nodeSize \cdot 4\ Byte & = 4^2 \cdot 10^6 Byte = 16 MB \\
\end{split}\end{equation}
(arrotondando e senza tenere conto di fattori secondari come i 32 byte occupati da overhead e puntatore).
\end{center}
Per la matrice invece possiamo assumere che, su una macchina a 64 bit, avrò il $57\%$ di righe a null ($0.57 \cdot 4\cdot 10^6 \cdot 8\ Byte = 18\ MB$) ed il 
restante $43\%$ di righe con mediamente 404 valori di relatedness codificati su 4 Byte ($0.43 \cdot 4 \cdot 10^6 \cdot 404 \cdot 4\ Byte = 2.8\ GB$). 

Quindi in totale potremmo stimare che solo questo dump occupa quasi 3 GB (notare che esiste un dump per ogni lingua supportata da Dandelion e che questo è uno 
dei dump più piccoli).

%%%%%%%%%%%%%%%%%%%%%%%%%%%%%%%%%%%%%%%%%%%%%%%%%%%%%%%%%%%%%%%%%%%%%%%%%%%%%%%%%%%%%%%%%%%%%%%%%%%%%%%%%%%%%%%%%%%%%%%%
\subsubsection{Stima della Matrice Completa}
Se allocassimo in memoria una matrice di Byte completa, di dimensioni $nodeSize \otimes nodeSize$, essa occuperebbe $(4 \cdot 10^6)^2 = 1.6 \cdot 10^{13}\ Byte = 16\ TB$.

Pur considerando che esistono strutture dati particolarmente ottimizzate per gestire array di grandi dimensioni con pochi valori al loro interno (teniamo presente che
più della metà dei valori nella matrice non sarebbero valorizzati), come SparseArray e SuperArrayList, bisogna tener presente che spesso, tali strutture dati, 
si basano su liste; in questo caso però difficilmente si otterrebbero prestazioni migliori di quelle dell'implementazione attuale dato che il tempo di lookup 
sarebbe sempre $O(lg_2(noseSize))$. 

In conclusione, a meno di uno spreco di memoria ulteriore, difficilmente si otterrebbero buoni risultati mantenendo come struttura dati di riferimento la matrice. 

%%%%%%%%%%%%%%%%%%%%%%%%%%%%%%%%%%%%%%%%%%%%%%%%%%%%%%%%%%%%%%%%%%%%%%%%%%%%%%%%%%%%%%%%%%%%%%%%%%%%%%%%%%%%%%%%%%%%%%%%
\subsubsection{Stima del problema inverso}
Una via praticabile potrebbe essere quella di vedere il problema al contrario, considerando che esistono solo 256 valori possibili di relatedness, 
per minimizzare lo spreco di memoria si potrebbe pensare all'allocazione di 256 array in cui si storicizzano tutte le coppie di id che hanno la stessa relatedness. 
In questo modo eliminerei la ridondanza costituita da valori uguali di relatedness salvati nel dump migliaia di volte.

Ci si accorge subito che anche questa via non è praticabile dato che per risparmiare 1 Byte (il valore della relatedness) dovrei allocarne altri 8 per le coppie di id (di tipo intero). 
Andrei ad occupare $0.43 \cdot 4 \cdot 10^6 \cdot 404 \cdot 2 \cdot 4 = 5.6 GB$, il doppio rispetto all'implementazione nativa, senza contare che 
le performance peggiorerebbero.

%%%%%%%%%%%%%%%%%%%%%%%%%%%%%%%%%%%%%%%%%%%%%%%%%%%%%%%%%%%%%%%%%%%%%%%%%%%%%%%%%%%%%%%%%%%%%%%%%%%%%%%%%%%%%%%%%%%%%%%%
\subsubsection{Stima implementazione di un grafo}
Si potrebbe pensare di implementare un grafo orientato, salvando per ogni nodo l'id e la lista di archi uscenti e su ogni arco la relatedness fra nodo di partenza 
e nodo di arrivo. Dovrei quindi usare una struttura dati del tipo:
 
\begin{lstlisting}[style=JavaStyle, caption=Struttura di un possibile grafo]
    public class Node{
        public int Id;
        public List<Arrow> Arrows;  
    }

    public class Arrow{
        public Node Node;
        public byte Relatedness;
    }
\end{lstlisting}

Tuttavia in questo modo sprecherei memoria perchè il puntatore al nodo successivo (ipotizzando di lavorare su una macchina a 64 bit) peserebbe più dell'identificativo 
del nodo stesso. Al che potrei sostituire il puntatore con un intero ma anche in questo caso avrei il dizionario $< Id,\ Relatedness >$ che pesa 5 Byte mentre nella 
postingList pesa 4 Byte perchè chiave e valore sono accorpati. Infine se seguissi la stessa politica di codifica del dizionario della postingList 
di fatto sarei tornato all'implementazione iniziale senza trarre alcun giovamento, se non lo svantaggio agguintivo di non poter accedere in O(1) ad un nodo arbitrario. 

%%%%%%%%%%%%%%%%%%%%%%%%%%%%%%%%%%%%%%%%%%%%%%%%%%%%%%%%%%%%%%%%%%%%%%%%%%%%%%%%%%%%%%%%%%%%%%%%%%%%%%%%%%%%%%%%%%%%%%%%
\section{Implementazione con Hash Map}
Sembra che l'unica via praticabile per mantenere tutti i dati in memoria, diminuendo la memoria occupata e massimizzando le prestazioni sia una funzione di Hash Map, 
che mappi gli id concatenati sul valore della relatedness; se la funzione fosse ben ottimizzata otterrei in O(1) la relatedness (guadagno prestazionale)
e potenzialmente potrebbe occupare meno memoria della matrice iniziale.

Purtroppo è molto difficile stimare a priori lo spazio occupato da un Hash Map, 
il modo più rapido è confrontare con dei benchmark l'implementazione nativa contro quella basata su Hash Map.

\subsection{Codice}
\begin{lstlisting}[style=JavaStyle, caption=Implentazione con HashMap]
public float Relatedness(int a, int b) throws Exception {
    if (a == b) {
        return 1f;
    }

    int nodeA = data.map[a];
    int nodeB = data.map[b];

    if (nodeA < 0 || nodeB < 0) {
        return 0f;
    }

    String key = (nodeA > nodeB) ? (nodeA + "." + nodeB) : (nodeB + "." + nodeA);
    try {
        byte value = (byte) data.hm.get(key);

        if (value == 0) {
            return 0f;
        } else {
            int byint = value + 128;
            float byteRel = byint / 255f;
            return data.configuredMinRel + byteRel * (1 - data.configuredMinRel);
        }
    } catch (Exception e) {
        return 0f;
    }
}

public RelatednessHashMap ReadDump(String dump) throws Exception {
    URL url = getClass().getResource(dump);
    File file = new File(url.getPath());
    BufferedReader fbr = new BufferedReader(new FileReader(file));
    String line = new String();

    try {
        RelatednessHashMap data = new RelatednessHashMap();
        int max_id = Integer.parseInt(fbr.readLine().trim());
        int nodesSize = Integer.parseInt(fbr.readLine().trim());
        data.configuredMinRel = Float.parseFloat(fbr.readLine().trim());
        data.configuredMinIntersection = Integer.parseInt(fbr.readLine().trim());
        data.threshold = Float.parseFloat(fbr.readLine().trim());

        data.map = new int[max_id + 1];
        data.hm = new HashMap();

        String END_OF_FILE = "" + '\0';

        while ((line = fbr.readLine()) != null) {
            if (line.trim().equals(END_OF_FILE.trim())) {
                break;
            }

            String[] splittedLine = line.split("\\s+");
            if (splittedLine.length != 3) {
                throw new Exception("Wrong format relatedness file for the line: " + line);
            }

            int wid = Integer.parseInt(splittedLine[0]);
            int node = Integer.parseInt(splittedLine[1]);
            data.map[wid] = node;

            if (!splittedLine[2].equals("null")) {
                byte[] postingList = Base64.getDecoder().decode(splittedLine[2].toString());

                int idx_value = ((postingList[0] & 0xFF) << 16)
                            + ((postingList[1] & 0xFF) << 8)
                            + ((postingList[2] & 0xFF) << 0);

                String key = idx_value + "." + node;
                data.hm.put(key, postingList[3]);
            }
        }

        return data;
    } catch (Exception e) {
        System.out.println(e.getMessage());
        return null;
    }
}
\end{lstlisting}

\subsection{Test}

\subsubsection{Generazione di un dump con dimensione arbitraria}
Per testare al meglio le varie implementazioni è stato creato un metodo per generare automaticamente un dump con dati random, della dimensione che si desidera;
il metodo \code{CreateDump()} accetta come parametri \code{maxId}, \code{nodeSize} e \code{nullPercentage} (la percentuale di righe con \code{postingList = null}) e genera: un dump 
analogo a quello reale, un dump con \code{postingList} non codificata in Base64 (utile per debug) e un file contenente una serie di test-cases della forma: 
\code{$<entityId; entityId; relatedness>$}.

Per essere più fedeli possibile all'originale le \code{postingList == null} sono uniformemente distribuite fra le righe (non tutte ammassate all'inizio o alla fine) e la lunghezza delle restanti 
è direttamente proporzionale al valore dell'id della riga (questo per poter avere casi di test su \code{postingList} di lunghezza massima e minima). 
Con questo metodo è anche molto facile stimare il numero di relazioni presenti nel dataset, basta calcolare l'area del triangolo che ha altezza e base pari a \code{nodeSize} (togliendo 
le \code{postingList null}).

\begin{center}
    \includegraphics[scale=0.45]{Sources/Img/c03_01.png}
    \captionof{figure}{Dump con struttura a triangolo}
\end{center}

\subsubsection{Test Hash Map}
Per testare le nuove implementazioni sono stati eseguiti gli stessi test-cases sullo stesso dump e messi a confronto tempi di risposta e occupazione di memoria della struttura dati.\\

\begin{tabular}{|l|l|l|l|l|}
    \hline
    \textbf{Type}    &   \textbf{MaxId - NodeSize}   &   \textbf{$\#$ test-cases}   &   \textbf{Read dump time}  &   \textbf{Testing time}            \\      
    \hline
    Native              &   1.000 - 900              &   1.000.000                  &   0.09s               &   0.90s                       \\
    HashMap             &   1.000 - 900              &   1.000.000                  &   0.17s               &   1.05s                       \\         
    \hline
    Native              &   10.000 - 9.000           &   1.000.000                  &   1.73s               &   1.16s                       \\
    HashMap             &   10.000 - 9.000           &   1.000.000                  &   20.24s              &   2.00s                       \\ 
    \hline
    Native              &   15.000 - 14.000          &   1.000.000                  &   4.91s               &   1.26s                       \\
    HashMap             &   15.000 - 14.000          &   1.000.000                  &   72.26s              &   2.42s                       \\ 
    \hline
    Native              &   20.000 - 19.000          &   1.000.000                  &   8.87s               &   1.32s                       \\
    HashMap             &   20.000 - 19.000          &   1.000.000                  &   133.32s             &   3.29s                       \\ 
    \hline
    Native              &   30.000 - 29.000          &   1.000.000                  &   19.71s              &   1.49s                       \\
    HashMap             &   30.000 - 29.000          &   1.000.000                  &   out of memory       &   out of memory               \\
    \hline                                    
\end{tabular}\\

Sono stati eseguiti vari test, a macchina scarica, con entrambe le implementazioni su dump di dimensione sempre crescente. 
Ogni test è stato replicato 10 volte e, con le medie dei risultati, è stata popolata la tabella sopra riportata.

L'implementazione nativa sembra superare sotto ogni aspetto l'implementazione con HashMap.

Come si può notare dalla tabella l'implementazione con HashMap impiega molto più tempo della nativa per leggere il dump e caricare in memoria la struttura dati;
questo non sembrerebbe un grosso problema perchè il tempo di lettura non influisce sulle performance dell'algoritmo potrebbe essere un problema se la funzione di crescita 
fosse esponenziale rispetto al numero di relatedness presenti nel dump. 

Anche se la funzione cresce molto velocemente, rappresentando i dati di test su un grafico, sembra sia comunque lineare (per il dump reale si stima che impieghi meno di un ora).

\begin{center}
    \includegraphics[scale=0.55]{Sources/Img/c03_02.jpg}
    \captionof{figure}{Funzione di crescita del tempo di lettura del dump al crescere del numero correlazioni fra entità (in rosso implementazione con HashMap, in blu implementazione nativa)}
\end{center}

Il grosso problema dell'implementazione con HashMap è lo spreco di memoria, l'ultimo test, con \code{nodeSize = 30.000}, è fallito perchè alla lettura del dump il container docker aveva allocato il 
78$\%$ della memoria della macchina (con 16GB di ram) contro il 13$\%$ utilizzato dall'implementazione nativa. 

Altra cosa inaspettata, al crescere del numero di correlazioni l'HashMap diventa così complessa da risolvere che, pur essendo stimato il tempo di look-up a $O(1)$, comunque supera il tempo 
di ricerca binaria ($O(lg_2(n)$)) con uno scostamento sempre crescente.\\

Stima del numero di correlazioni nel dump originale: 
\begin{equation}\begin{split}
    49.22289 \cdot (4922289 \cdot 0,000101) \cdot 0.43 = 1.041.843.947\\
\end{split}\end{equation}

Stima del numero di correlazioni con \code{nodeSize = 29.000}; 
\begin{equation}\begin{split}
    29.000 \cdot 29.000 \cdot 0.5 \cdot 0.43 = 180.815.000 \\
\end{split}\end{equation}

Nonostante ci sia un ordine di grandezza di differenza fra il numero di relatedness storicizzate nel dump originale e quello nel dump autogenerato (con \code{maxId = 30000} e \code{nodeSize = 29.000}), 
l'implementazione con HashMap risulta estremamente inefficiente rispetto a quella nativa. 
      \newpage     
      \chapter{StrepHit}
\label{cha:strephit}
La terza parte del progetto consiste nell'implementazione di uno script in Python (versione 2.7) che, dato in input un dataset di QuickStatements, deve restituire in output il dataset
arricchito con nuovi riferimenti.

Ogni riga del dataset di input presenta una proprietà \code{P854} (reference URL~\cite{P854}) con abbinata l'URL di riferimento della risorsa; 
lo script deve eseguire una query SPARQL per trovare l'item nel knowledge-base di Wikidata che corrisponde/identifica il dominio della URL. 
Se questo item di Wikidata esiste allora si cerca anche una e la proprietà di tale item che definisca uno schema\footnote{
    \textbf{Schema della URL}, in Wikidata esiste una proprietà \code{P1630} (formatter URL~\cite{P1630}) che definisce un template generale per una categoria di URL. 
    Per esempio una \textit{formatter URL} può essere la seguente: \code{http://www.nndb.com/people/$\$$1/} dove $\$1$ è un placeholder che rappresenta una qualsiasi stringa.  
} valido per l'URL in questione.

\section{Esempio di funzionamento}
Per esempio dato il seguente Quickstatement in input:
\begin{lstlisting}[style=QuickstatementsStyle, caption=Riga del dump]
    Q193660	P106	Q207628	S854	"http://www.nndb.com/people/031/000097737/"
\end{lstlisting}

Notiamo una prima relazione semantica 
\code{Q193660} (Ramon Llull~\cite{Q193660}) 
\code{P106} (occupation~\cite{P106}) 
\code{Q207628} (musical composition~\cite{Q207628}) 
che ci dice semplicemente che \qt{Ramon Llull lavora come compositore musicale}.

Segue la proprietà, di maggiore interesse, \code{S854 "http://www.nndb.com/people/031/000097737/"} che indica la provenienza dell'informazione. 

Lo script procede estrapolando il dominio dalla reference url (\code{www.nndb.com}) e, con una query SPARQL, cerca un item di riferimento per tale dominio in Wikidata 
(non è sempre garantita la presenza).

In questo esempio l'item di riferimento è \code{Q1373513} (NNDB~\cite{Q1373513}) perchè presenta la proprietà \code{P856} (official website~\cite{P856}) che corrisponde al dominio cercato;
lo stesso item ha anche una proprietà chiamata \qt{Wikidata property} (\code{P1687}) con abbinato l'identificativo di un'altra proprietà chiamata  
\qt{NNDB people ID} (\code{P1263}).

Se andiamo ad analizzare la proprietà \qt{NNDB people ID} (\code{P1263}~\cite{P1263}) notiamo che presenta a sua volta una proprietà 
\qt{formatter URL} (\code{P1630}~\cite{P1630}) il cui valore è \code{http://www.nndb.com/people/$\$1/$}. 
Il valore finale \code{$\$1$} nella \textit{formatter URL}\rq\rq\ è un placeholder che sta ad indicare la parte della URL che corrisponde al valore della proprietà prescelta. 

In questo esempio avremmo:
\begin{lstlisting}[style=QuickstatementsStyle, caption=Esempio di formatter URL]
    URL originale presente nel dump:        http://www.nndb.com/people/031/000097737/
    Formatter URL:                          http://www.nndb.com/people/$1/
    Valore della proprieta':                031/000097737
    (estrapolato grazie al placeholder $1)
\end{lstlisting}

Lo script andrà quindi ad estrapolare dalla URL di partenza la stringa \code{031/000097737} che corrisponde al valore della proprietà \code{P1263} e andrà ad arricchire il dump 
con questa informazione aggiuntiva.

\begin{lstlisting}[style=QuickstatementsStyle, caption=Risultato dello script]
    Q193660	P106	Q207628	S854	"http://www.nndb.com/people/031/000097737/"	S248	Q1373513	S1263	"031/000097737"	S813	2018-06-04T02:19:10Z/14
\end{lstlisting}

I riferimenti aggiuntivi sono: 
\code{S248} (stated in~\cite{P248})
\code{Q1373513} (NNDB~\cite{Q1373513})
\code{S1263} (NNDB people ID~\cite{P1263})
\code{"031/000097737"} (il valore estrapolato, people ID)
\code{S813} (retrieved~\cite{P813})
\code{2018-06-04T02:19:10Z/14} (timestamp).

\section{SPARQL Query}
Per risolvere ogni dominio e ogni \qt{formatter URL} sconosciuta si usa una sola query; si è cercato di limitare il più possibile il numero di query effettuate 
(dato che alcune query possono impiegare svariati secondi per essere eseguite), 
salvando in memoria e su disco i risultati di quelle già lanciate in precedenza per minimizzare il tempo di computazione dello script.

\begin{lstlisting}[style=SPARQLStyle, caption=SPARQL query per cercare item e proprietà relativi al dominio \code{"www.nndb.com"}]
    select Distinct ?subjects ?wikidataProperty ?formatterUrlLabel ?sitelinkLabel
    where {
        {
            BIND("www.nndb.com" AS ?domain).
            SERVICE wikibase:label { bd:serviceParam wikibase:language "[AUTO_LANGUAGE],en". }
            ?subjects wdt:P856 ?sitelink ;
                      wdt:P1687 ?wikidataProperty.
            ?wikidataProperty wdt:P1630 ?formatterUrl
            FILTER (REGEX(str(?formatterUrl), ?domain) || REGEX(str(?sitelink), ?domain)).
        }
        union
        {
            BIND("www.nndb.com" AS ?domain).
            SERVICE wikibase:label { bd:serviceParam wikibase:language "[AUTO_LANGUAGE],en". }
            ?subjects wdt:P856 ?sitelink ;
            FILTER REGEX(str(?sitelink), ?domain).
        }
    }
\end{lstlisting}

Va premesso che esistono molti modi per ottenere lo stesso risultato, con costrutti molto meno verbosi, tuttavia questa è l'unica query trovata che attualmente non manda in timeout l'endpoint.

Il superamento del timeout è sicuramente dovuto al fatto che internamente si usano delle regular expressions che appesantiscono molto l'esecuzione della query, soprattutto su 
grandi knowledge-base come quello di Wikidata. 

In SPARQL usare una ricerca per stringa è certamente una forzatura perchè solitamente si conoscono già a priori item e propery che interessano tuttavia, in questo caso, è 
stato necessario adottare la ricerca per regular expression dato che lo script deve proprio affrontare il problema inverso.

La query è il risultato dell'unione di due sub-query; la prima sub-query cerca tutte le proprietà che posseggono una proprietà \qt{formatter URL} (\code{P1630}) il cui valore ha come dominio quello del 
Quickstatement che lo script sta analizzando (in questo caso \code{"www.nndb.com"}); in oltre controlla che tale priprietà sia legata ad un item il cui \qt{official website} (\code{P856}) 
sia coerente con il dominio in questione. 

La seconda sub-query invece cerca tutti gli item il cui \qt{official website} (\code{P856}) abbia lo stesso dominio di quello del Quickstatement che lo script sta analizzando. 
Questa query è necessaria perchè non sempre esiste una proprietà la cui \qt{formatter URL} sia coerente con il link che si sta analizzando; può succedere che esista solo l'item 
relativo al database in questione ma non la proprietà specifica, in pochi casi non esiste nemmeno tale item. 

In alcuni casi può succedere che l'item relativo al database esista ma abbia un dominio completamente differente da quello della proprietà la cui \qt{formatter URL}
presenta un match con l'URL in analisi. 

La difficoltà principale nella realizzazione dello script è stata proprio quella di gestire una moltitudine di casi particolari, derivanti dal fatto che il dataset è molto grande
(più di 500.000 Quickstatements) ed eterogeneo (presenta svariati domini differenti e qualche URL completamente sbagliata o deprecata). 

\begin{center}
    \includegraphics[width=\linewidth]{Sources/Img/c04_01.png}
    \captionof{figure}{Risultato della query per il dominio \code{"www.nndb.com"}}
\end{center}

\section{Lo script}

\subsection{Struttura del progetto}
Il progetto presenta una cartella \qt{assets} contenente tutti i file di input e output, i \code{.json} di configurazione e i \code{.log} degli errori; 
nella cartella \qt{business} sono contenuti i servizi, i metodi di utilità e le query. 

La cartella \qt{domain} contiene i modelli e le localizations (nel file \code{localizations.py} sono definite anche una serie di costanti da settare a seconda 
delle esigenze per configurare lo script). 

In \qt{tests} abbiamo tutte le classi di test basate sulla libreria unittest~\cite{unittest}; come per il Client $C\#$ anche in questo caso è stato configurato Travis per far eseguire i test 
automaticamente ad ogni pull-request. 

Infine nella root del progetto abbiamo l'entry point dello script (\code{main.py}) e l'entry point dei test (\code{test.py}), i requirement per il package manager 
e il file di configurazione \code{strephit.py}, nel caso si voglia lanciare lo script in un virtualenv~\cite{virtualenv} tramite la libreria Click~\cite{click}.

\subsection{Algoritmo}
Istanziando l'oggetto \code{QuickStatementsService} vengono caricati automaticamente in memoria i \code{mappings} presenti nella cartella \qt{assets}; 
i \code{mappings} sono dei file \code{.json} in cui lo script salva i risultati delle le chiamate, all'endpoint SPARQL, già effettuate.

\begin{lstlisting}[style=jsonStyle, caption=Some Code]
    "www.nndb.com": [
        {
            "db_id": "Q1373513", 
            "db_property": "S1263", 
            "to_upper_case": false, 
            "url_pattern": "http://www.nndb.com/people/$1/"
        }
    ], 
\end{lstlisting}

Il metodo \code{add\u db\u references\u async()} cicla su ogni riga del dataset e per ogniuna di esse chiama un handler 
(\code{add\u db\u references\u async\u handler()}), passandogli un oggetto \code{QuickStatement} contenente tutte le informazioni della riga, 
che procede con l'analisi dell'URL e la generazione dei riferimenti mancanti.

A questo punto, in modo sincrono o asincrono, a seconda del settaggio delle costanti in \code{localizations.py} (\code{IS\u ASYNC \u MODE = True/False}), 
viene chiamato il metodo \code{generate\u db\u reference()} che provvede a controllare che nei \code{mappings} ci sia già una entry con formatter URL compatibile 
con l'URL del Quickstatement e a generare i riferimenti mancanti.

Se nei \code{mappings} non esiste nessuna entry compatibile con l'URL viene chiamato il metodo \code{new\u mapping()} che provvede a chiamare l'endpoint SPARQL 
e a selezionare il risultato migliore per aggiungerlo ai \code{mappings}. 

Se la costante \code{MAP\u ALL\u RESPONSES} viene settata a \code{True}, ogni risposta del endpoint SPARQL viene salvata interamente nei \code{mappings};
questo previene qualsiasi altra chiamata all'endpoint per un determinato dominio ma accresce la dimensione dei \code{mappings} a dismisura, rendendo la ricerca più lenta.
 
\subsection{Refresh delle URL}
Nel dataset analizzato alcuni domini sono deprecati tant'è che tentando di accederci con un browser si viene reindirizzati ad altri domini oppure si passa da http ad https.

Per non perdere questi riferimenti si è deciso di implementare anche una funzionalità che data una lista di domini va a fare un \qt{refresh} di tutte le URL aventi tali domini 
e ad eliminare le righe contenenti URL inesistenti.

A fine procedura vengono salvati i \code{.log} con la lista delle URL modificate e delle righe eliminate dal dataset.


      \newpage     
      \chapter{Conclusioni}
Personalmente ritengo che l'esperienza di tesi sia stata molto positiva in quanto ho avuto la possibilità di entrare in contatto con diverse realtà lavorative fuori dal comune; 
sotto il profilo tecnico questo progetto mi ha portato ad apprendere molte nuove tecnologie che non conoscevo e a rafforzare conoscenze pregresse.

La prima parte, se pur non particolarmente complessa, è stata interessante perchè riguardava l'intero ciclo di vita di una libreria, dall'implementazione alla pubblicazione;
mi ha dato modo di rafforzare le nozioni di programmazione in C$\#$ e di seguire, per la prima volta, il deploy di una libreria su Nuget. 
La libreria è funzionante e ad oggi conta un centinaio di download, risultato sicuramente positivo.

La seconda parte si è rivelata, alla fine, un'analisi fine a se stessa, non si sono ottenuti risultati utili per poter ottimizzare l'algoritmo esistente;
nonostante ciò la ritengo personalmente un'esperienza positiva dato che mi ha permesso di provare svariate nuove tecnologie molto utili, prima fra tutte Docker. 

Mi ha colpito molto l'elevata complessità insita nel problema della valutazione prestazionale di un processo; 
infatti nel corso dell'analisi sono stati provati svariati javaagent, per monitorare il processo, che regolarmente risultavano inattendibili, 
in parte per l'interazione con altri processi e in parte per il ridotto scope di visibilità del software di monitoraggio stesso 
(senza contare la presenza di svariati fattori aleatori come l'azione del Garbage Collector).

Altra cosa che mi ha sorpreso molto è stato l'incredibile calo prestazionale della funzione hash all'aumentare delle dimensioni del dominio, tanto che la computazione dell'hash stesso 
richiedeva più tempo della ricerca binaria.

Penso sia comunque possibile trovare un'implementazione migliore dell'attuale; sarebbe un interessante spunto di ricerca ulteriore valutare se sia possibile creare una 
funzione hash ad hoc per questo problema; sarebbe anche interessante valutare implementazioni basate su strutture dati salvate su disco anzichè in memoria 
(database, indici di Lucene\cite{lucene}, ecc.), anche se sicuramente i tempi di risposta aumenterebbero di qualche ordine di grandezza per il solo accesso al disco. 

La terza parte mi ha permesso di conoscere il mondo di Wikidata e del semantic web più in generale, dandomi l'opportunità di contribuire, per la prima volta, 
ad un progetto open source così rilevante.

Si è ottenuto il risultato sperato, in quanto lo script funziona correttamente, tuttavia sarebbe interessante riuscire ad ottimizzare lo script visto che attualmente, 
per computare un dataset da 500.000 righe, impiega qualche ora; 
va da sè che spesso si hanno ritardi, anche di svariati secondi, totalmente indipendenti dallo script (connessione ad internet, tempo di attesa per le query SPARQL 
e per il "refresh" delle URL deprecate, tramite chiamate HTTP); tolti questi punti sicuramente si possono ottimizzare le routine maggiormente usate nello script 
per abbassare i tempi di computazione.

In conclusione voglio ringraziare tutti coloro che mi hanno aiutato e seguito in questo progetto di tesi.
      \newpage     
    \endgroup

    % bibliografia in formato bibtex
    
    \addcontentsline{toc}{chapter}{Bibliografia} % aggiunta del capitolo nell'indice
    \bibliographystyle{plain} % stile con ordinamento alfabetico in funzione degli autori
    \bibliography{biblio}
    %% Nella bibliografia devo avere tutte le fonti consultate; va redatta in ordine alfabetico sul cognome del primo autore. 
    %% LIBRI
    %% Cognome e iniziale del nome autore, la data di edizione, titolo, casa editrice, eventuale numero dell’edizione. 
    %% SITOGRAFIA
    %% Lista di indirizzi Web disposti in ordine alfabetico. 
    %% URL, cognome e nome dell’autore, il titolo, sottotitolo, data di ultima consultazione della risorsa (gg/mm/aaaa).
    \titleformat{\chapter}
        {\normalfont\Huge\bfseries}{Allegato \thechapter}{1em}{}
    % sezione Allegati - opzionale
    \appendix
    \input{Chapters/attachments.tex}
\end{document}
